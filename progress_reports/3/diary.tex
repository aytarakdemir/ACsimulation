% !TEX TS-program = pdflatex
% !TEX encoding = UTF-8 Unicode

% This is a simple template for a LaTeX document using the "article" class.
% See "book", "report", "letter" for other types of document.

\documentclass[11pt]{article} % use larger type; default would be 10pt

\usepackage[utf8]{inputenc} % set input encoding (not needed with XeLaTeX)

%%% Examples of Article customizations
% These packages are optional, depending whether you want the features they provide.
% See the LaTeX Companion or other references for full information.

%%% PAGE DIMENSIONS
\usepackage{geometry} % to change the page dimensions
\geometry{a4paper} % or letterpaper (US) or a5paper or....
% \geometry{margin=2in} % for example, change the margins to 2 inches all round
% \geometry{landscape} % set up the page for landscape
%   read geometry.pdf for detailed page layout information

\usepackage{graphicx} % support the \includegraphics command and options

\usepackage[parfill]{parskip} % Activate to begin paragraphs with an empty line rather than an indent

%%% PACKAGES
\usepackage{booktabs} % for much better looking tables
\usepackage{array} % for better arrays (eg matrices) in maths
\usepackage{paralist} % very flexible & customisable lists (eg. enumerate/itemize, etc.)
\usepackage{verbatim} % adds environment for commenting out blocks of text & for better verbatim
\usepackage{subfig} % make it possible to include more than one captioned figure/table in a single float
% These packages are all incorporated in the memoir class to one degree or another...

%%% HEADERS & FOOTERS
\usepackage{fancyhdr} % This should be set AFTER setting up the page geometry
\pagestyle{fancy} % options: empty , plain , fancy
\renewcommand{\headrulewidth}{0pt} % customise the layout...
\lhead{}\chead{}\rhead{}
\lfoot{}\cfoot{\thepage}\rfoot{}

%%% SECTION TITLE APPEARANCE
\usepackage{sectsty}
\allsectionsfont{\sffamily\mdseries\upshape} % (See the fntguide.pdf for font help)
% (This matches ConTeXt defaults)

%%% ToC (table of contents) APPEARANCE
\usepackage[nottoc,notlof,notlot]{tocbibind} % Put the bibliography in the ToC
\usepackage[titles,subfigure]{tocloft} % Alter the style of the Table of Contents
\renewcommand{\cftsecfont}{\rmfamily\mdseries\upshape}
\renewcommand{\cftsecpagefont}{\rmfamily\mdseries\upshape} % No bold!

%%% END Article customizations

%%% The "real" document content comes below...

\title{Thesis Progress Report 3}
\author{Aytar Akdemir \\ 150170115}
%\date{} % Activate to display a given date or no date (if empty),
         % otherwise the current date is printed 

\begin{document}
\maketitle

\section{Superviser Input - 17.11.2021}

\subsection{Multilevel Security Review}

It was defined in the Rainbow series published by the United States Department of Defense.
Trusted Computer System Evaluation Criteria that i found belongs in the Rainbow series.
	
What MLS essentially is the intersection of the Biba Integrity Model and the Bell-LaPadula Confidentiality Model.
This leaves us with a system in which every subject can only access to its security level.
There is no communication between security levels.
	
To perform that communication without leaving the system open to attacks, Policy Enfrocement Points are used between every security level.
We assume that the PEPs are completely secure. The reason for privilege escalation is the need for communication between the security levels.

Even though we assumed PEP is completely secure, the programs used in the Operationg System generally have vulnerabilities.
For example, a program might perform privilege escalationin order to function.
If an attacker gains access to said program, they will access to the higher security level.
In out simulation, we will assume that the system damage is caused by using insecure programs.

\textbf{Insecure Program Example:} On Linux, a daemon mail client named Sendmail had root access and it worked continuously on the background.
In a case of intrusion to the Sendmail, the attacker might gain root privileges.

\subsection{Formalizations}

Subjects access to the objects using processes.
Subject will transfer its privileges to the process it intends to run.
When the process is trying to access to the subjects, OS applies PEP to the process.
Newer OSes have their own PEPs while older OSes need PEPs developed on top of them.

We talked about four possible formalization techniques for the simulation.

\begin{itemize}
\item Clark Wilson - Graph based
\item Take-Grant - Graph based
\item State Machine - It allows certain states.
\item Petri Net
\end{itemize}

Main problem with the graph based formalizations is the complexity.
In order to compute all the access possibilites we also need to sacrifice a big chunk of memory.
We need to use a combination to find a formalization technique with a reasonable complexity.
Penetration testing that I mentioned in the literature review might be a key to this.
Fuzzer is another option for detecting vulnerabilities.


\end{document}
