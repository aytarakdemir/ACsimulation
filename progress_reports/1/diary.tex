% !TEX TS-program = pdflatex
% !TEX encoding = UTF-8 Unicode

% This is a simple template for a LaTeX document using the "article" class.
% See "book", "report", "letter" for other types of document.

\documentclass[11pt]{article} % use larger type; default would be 10pt

\usepackage[utf8]{inputenc} % set input encoding (not needed with XeLaTeX)

%%% Examples of Article customizations
% These packages are optional, depending whether you want the features they provide.
% See the LaTeX Companion or other references for full information.

%%% PAGE DIMENSIONS
\usepackage{geometry} % to change the page dimensions
\geometry{a4paper} % or letterpaper (US) or a5paper or....
% \geometry{margin=2in} % for example, change the margins to 2 inches all round
% \geometry{landscape} % set up the page for landscape
%   read geometry.pdf for detailed page layout information

\usepackage{graphicx} % support the \includegraphics command and options

\usepackage[parfill]{parskip} % Activate to begin paragraphs with an empty line rather than an indent

%%% PACKAGES
\usepackage{booktabs} % for much better looking tables
\usepackage{array} % for better arrays (eg matrices) in maths
\usepackage{paralist} % very flexible & customisable lists (eg. enumerate/itemize, etc.)
\usepackage{verbatim} % adds environment for commenting out blocks of text & for better verbatim
\usepackage{subfig} % make it possible to include more than one captioned figure/table in a single float
% These packages are all incorporated in the memoir class to one degree or another...

%%% HEADERS & FOOTERS
\usepackage{fancyhdr} % This should be set AFTER setting up the page geometry
\pagestyle{fancy} % options: empty , plain , fancy
\renewcommand{\headrulewidth}{0pt} % customise the layout...
\lhead{}\chead{}\rhead{}
\lfoot{}\cfoot{\thepage}\rfoot{}

%%% SECTION TITLE APPEARANCE
\usepackage{sectsty}
\allsectionsfont{\sffamily\mdseries\upshape} % (See the fntguide.pdf for font help)
% (This matches ConTeXt defaults)

%%% ToC (table of contents) APPEARANCE
\usepackage[nottoc,notlof,notlot]{tocbibind} % Put the bibliography in the ToC
\usepackage[titles,subfigure]{tocloft} % Alter the style of the Table of Contents
\renewcommand{\cftsecfont}{\rmfamily\mdseries\upshape}
\renewcommand{\cftsecpagefont}{\rmfamily\mdseries\upshape} % No bold!

%%% END Article customizations

%%% The "real" document content comes below...

\title{Thesis Journal \\ Week 1}
\author{Aytar Akdemir}
%\date{} % Activate to display a given date or no date (if empty),
         % otherwise the current date is printed 

\begin{document}
\maketitle

\section{Types of Access Control}

First step is to determine the types of AC systems to be simulated. These could be merged, removed. Other AC systems might be added. In this section, a simple description of the AC systems are specified. The goal of this part is to determine which of the specified AC systems are suitable to model and simulate.

\subsection{Mandatory Access Control}

Subjects and objects have security attributes. When a subject tries to access an object, the operation is tested if it complies with the policy. Operationg system enforces the policy rules.

Security Policy Administrator controls the policies, users cannot change policy.

\subsection{Discretionary Access Control}
Access control is determined by the group which the subjects belong to.

Subjects can transfer permissions to other subjects.

\subsection{Role-based Access Control}

Roles are created for different job functions and assigned to users. Permissions come with the role and there is no need for the micromanagement of the permissions.


\subsection{Identity-based Access Control}

A user can only access to certain resources if their identity can be matched. A mechanism to authenticate the identity of the user is required. (e.g. Biometric readings)


\subsection{Attribute-based Access Control}

Uses policy that use attributes. Attributes may be:
\begin{itemize}		
\item User attributes (Includes role)
\item Resource attributes
\item Object attributes
\item Environment attributes
\end{itemize}

\subsection{Organization-based Access Control}
Focuses on Subject - action - object 
\begin{itemize}		
\item Roles: Subjects are abstracted into roles
\item Activity: Actions are grouped to comply to the same rules
\item View: Set of objects with the same security rule
\end{itemize}
Every security model is defined by an organization and is parametrized by it. The model includes permissions, obligations and prohibitions.


\subsection{Lattice-based Access Control}

A lattice is used to define the security levels of objects and clearance levels of the subjects. A subject can only access an object if the clearance level is higher than the security level of the object.


\subsection{Graph-based Access Control}
The access rights are granted to objects but also accounts. Difference from the  role-based access control is GBAC uses a organizational query language for defining access rights while RBAC uses enumeration.

Two building blocks
\begin{itemize}		
\item A semantic graph modeling an organization
\item A query language
\end{itemize}


\subsection{History-based Access Control}

Run-time rights of a code is determined by examining the attributes of the code and the requests that were made by that code.


\section{Modelling}

After determining which ones will be used, all AC systems should be modelled using flowcharts. All flowcharts should be standardized for comparison. These flowcharts will be used as guides when programming the simulation.


\end{document}
